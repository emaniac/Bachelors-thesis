\documentclass{article}

\usepackage{color}
\usepackage{amsmath}
\usepackage{subcaption}
\captionsetup{compatibility=false}
\usepackage{graphicx}
\usepackage{grffile}
\usepackage{epstopdf}
\begin{document}

Mam takovy pocit, ze kdyz jsi rozsiroval ten system pridavanim toho odhadu chyby z kalmana, vnes sis tam zpozdeni. Tady je muj postup myslenek a nastinene reseni, jak to popsat pomoci jenom 4 stavu. 
\section{tvuj jednoduchy system}

\begin{equation}
\begin{split}
\mathbf{A}_{x, y} = \begin{bmatrix}
1 & 0.0114 & 0\\
0 & 1 & 0.0114\\
0 & 0 & P_1
\end{bmatrix}, \mathbf{B}_{x, y} = \begin{bmatrix}
0\\
0\\
P_0
\end{bmatrix},
\end{split}
\label{eq:attitude_LTI_big_identified}
\end{equation}

To znamena, ze stavova rovnice je 

\begin{equation}
x_{3, t+1} = P_1 x_{3, t}+P_0u_t
\end{equation}
kde $x_{3}$ je zrychleni. Kdybych modeloval tu poruchu, tak se budu snazit dostat rovnici

\begin{equation}
\label{eq:aaa}
x_{3, t+1} = P_1 x_{3, t}+P_0u_t + e
\end{equation}

kde za $e$ povazuju nejakou 'konstantni' odhadnutou poruchu.

\section{tvuj rozsireny system}
\begin{equation}
\begin{split}
\mathbf{A}_{x, y} = \begin{bmatrix}
1 & 0.0114 & 0 & 0 & 0 \\
0 & 1 & 0.0114 & 0 & 0\\
0 & 0 & 0 & 1 & 1 \\
0 & 0 & 0 & P_1 & 0 \\
0 & 0 & 0 & 0 & 1
\end{bmatrix}, \mathbf{B}_{x, y} = \begin{bmatrix}
0\\
0\\
0\\
P_0\\
0
\end{bmatrix},
\end{split}
\label{eq:attitude_LTI_big_identified}
\end{equation}

Kdybych to mel popsat stavovou rovnici, dostanu
\begin{equation}
\begin{split}
x_{3, t+1} = x_{4, t} + x_{5, t} \\
x_{4, t+1} = P_1 x_{4, t}+P_0 u_t \\
x_{5, t+1} = x_{5, t} = porucha \;\;e\\
Po \;\;substituci \;\;dostanu \\
x_{3, t+1} = P_1 x_{4, t-1}+P_0 u_{t-1} + x_{5, t-1}
\end{split}
\end{equation}
Tim chci ukazat, ze se do toho stavu promitnou vstupy z predpredchoziho kroku a ne z predchoziho jako v \ref{eq:aaa}. Na zpozdeni poruchy snad nezalezi, jestli se meni pomalu. Nastesti tohle zpozdeni snad nema uplne moc velky vyznam v realnem rizeni

\section{Muj navrhovany system}

Nastavit poruchu na stav $x_4 = e$ a pouzit tyhle matice

\begin{equation}
\begin{split}
\mathbf{A}_{x, y} = \begin{bmatrix}
1 & 0.0114 & 0 & 0\\
0 & 1 & 0.0114 & 0\\
0 & 0 & P_1 & 1\\
0 & 0 & 0 & 1
\end{bmatrix}, \mathbf{B}_{x, y} = \begin{bmatrix}
0\\
0\\
P_0\\
0
\end{bmatrix},
\end{split}
\end{equation}

Tim bys mohl dostat tyhle stavove rovnice:

\begin{equation}
\begin{split}
x_{3, t+1} = P_1 x_{3, t} + x_{4, t} +P_0 u_t \\
x_{4, t+1} = x_{4, t} = porucha \;\;e\\
Po \;\;substituci \;\;dostanu \\
x_{3, t+1} = P_1 x_{3, t}+P_0 u_{t} + x_{4, t-1}
\end{split}
\end{equation}

Timhle jsem se zbavil jednoho stavu a toho jedno krokoveho zpozdeni. Zpozdeni odhadu chyby mi zustalo, ale to by snad nemuselo tolik vadit. 

Snad to dava smysl, mozna jsem neco prehlidnul a je to cely kravina :D


\end{document}